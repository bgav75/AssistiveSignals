\documentclass[10pt]{article}
\usepackage[utf8]{inputenc}
\usepackage[english]{babel}
\usepackage{amsfonts}
\usepackage{amssymb}
\usepackage{xcolor}
\usepackage{subfigure}
\usepackage{graphicx}
\usepackage{comment}
\usepackage{amsmath}
\usepackage{enumitem}
\usepackage{setspace}
\onehalfspacing
\usepackage{fullpage}
\usepackage{hyperref}
\usepackage{natbib}
\bibliographystyle{apalike}
\usepackage{amsthm}

\newcommand{\R}{\mathbb{R}}
\newcommand{\X}{\mathcal{X}}
\newcommand{\Y}{\mathcal{Y}}
\newcommand{\E}{\mathbb{E}}


\title{Learning From Mistakes}
\author{Bryce McLaughlin}

\begin{document}

\section*{Learning From Mistakes}

The practices of designing a prediction for the uses of automation and assistance are often confused as being identical. When a prediction is used for automation it acts as a decision rule: turning observable inputs into a final decision which is then put into practice. In contrast, an assistive prediction attempts to provide information to an autonomous decision-maker in hopes that they will prevent the decision-maker from committing would-be errors without eroding their properly taken decisions. In this way a decision rule wants to act as a substitute for the information a decision-maker stores while the assistive prediction wishes to acts a complement. This distinction suggests different knowledge basis from which the prediction should build.

While a decision rule needs to perform well (or at least well on average) over the entirety of a domain to be considered a success, the information contained in an assistive rule only needs to switch the decision-makers mistaken choices to accurate ones. By identifying a population over which the decision-maker commonly makes mistakes and limiting interventions to those regions, an assitive tool raises the fraction of opportunities it engages with where it has potential to make a positive impact. Similarly by building off problem instances similar to those the decision-maker was unable to perform on, the decision rule builds on a complementary set of knowledge, in a sense learning the part of the problem the decision-maker does not understand. 

\subsection*{Model}



\subsection*{Experiment}

\begin{enumerate}
	\item Participants will evaluate a set of 25 profiles from the dataset $X$ making decisions $Y$ on them.
	\item Based on the differences between the relationship between the true outcome of 
	\item
\end{enumerate}

\end{document}